\documentclass[dvipdfmx, 10.5pt]{beamer}
%%%% 和文用 %%%%%
\usepackage{bxdpx-beamer}
\usepackage{pxjahyper}
\usepackage{minijs}%和文用
\renewcommand{\kanjifamilydefault}{\gtdefault}%和文用
\usepackage{comment} %コメントアウト用

%%%%%%%%%%%%%%%%%%%%%%%%%%
%% usepackage 群
%%%%%%%%%%%%%%%%%%%%%%%%%%
\usepackage{amsmath,bm} %多次元空間ベクトルRを表記するのに必要
\usepackage{amsfonts}
\usepackage{ascmac} %枠付き文章を表記するのに必要
\usepackage{amssymb}
% \usepackage[dvipdfmx]{animate} %アニメーション
% \usepackage[dvipdfmx]{graphicx} %画像挿入
% \mathbb{R}^{l} %表記例
\usepackage{algorithm}
% \usepackage{algorithmicx}
\usepackage{algpseudocode}
\usepackage{animate} %アニメーション

%%%%%%%%%%%%%%%%%%%%%%%%%%
%% bibtex 群
%%%%%%%%%%%%%%%%%%%%%%%%%%
% biblatex.標準のバックエンドはbiber
\usepackage{biblatex}
% bibtexのファイル
% \addbibresource{mybib.bib}
% 参考文献のインデックスを文字列で表示
% \setbeamertemplate{bibliography item}[text]

%%%%%%%%%%%%%%%%%%%%%%%%%%
%% tikz 群
%%%%%%%%%%%%%%%%%%%%%%%%%%
\usepackage{tikz}
\usetikzlibrary{positioning}

\usetikzlibrary{calc, intersections, arrows, math}
\usetikzlibrary{arrows, positioning, shapes.callouts}
\usetikzlibrary{decorations,decorations.pathreplacing}

\graphicspath{{./fig/}}

%%%% スライドの見た目 %%%%%
\usetheme{Madrid}
\usefonttheme{professionalfonts}

%%%% metropolisの設定 %%%%%
% \metroset{block=fill}


%\useoutertheme[subsection=false]{smoothbars}%ヘッダーにセクション表示
\useinnertheme{circles} % 箇条書きをシンプルに

\setbeamercovered{transparent}%消えている文字をうっすらと表示
\setbeamertemplate{footline}[frame number]%フッターをページ番号だけに
\setbeamerfont{footline}{size=\scriptsize}%ページ番号小さく
\setbeamerfont{frametitle}{size=\large}%フレームタイトルちょい小さく
\setbeamercolor{footline}{bg=black}%ページ番号を太く
\setbeamersize{text margin left=.75zw, text margin right=.75zw}%スライドの横の空白を調節
% \setbeamertemplate{itemize item}{\small\raise0.5pt\hbox{$\bullet$}}
\setbeamertemplate{itemize subitem}{\scriptsize\raise1.5pt\hbox{$\blacktriangleright$}}
\setbeamertemplate{itemize subsubitem}{\scriptsize\raise1.5pt\hbox{$\bigstar$}}

\setbeamertemplate{enumerate items}[default]%enumerate環境のitemを見やすくする
\setbeamertemplate{section in toc}[square]%outlineのボールを四角に
\setbeamertemplate{navigation symbols}{}%右下のアイコンを消す

% blockの色定義
\definecolor{BlueTOL}{HTML}{222288}
\definecolor{BrownTOL}{HTML}{666633}
\definecolor{GreenTOL}{HTML}{228822}
\definecolor{RedTOL}{HTML}{882222}
\definecolor{myorange}{HTML}{FF8C00} % 彩度や明度の低いオレンジ(強調色で用いる)

\setbeamercolor{block title}{fg=white, bg=BlueTOL}
\setbeamercolor{block body}{fg=black, bg=BlueTOL!10!white}
\setbeamercolor{block title alerted}{fg=white, bg=RedTOL}
\setbeamercolor{block body alerted}{fg=black, bg=RedTOL!10!white}
\setbeamercolor{block title example}{fg=white, bg=GreenTOL}
\setbeamercolor{block body example}{fg=black, bg=GreenTOL!10!white}

%%%%

%%%%%%%%%%%%%%%%%%%%%%%%%%%%%%%%%%%%%%%%%%%%
%%いろいろ便利なもの
%%%%%%%%%%%%%%%%%%%%%%%%%%%%%%%%%%%%%%%%%%%%
\usepackage{here} %[hbtp]の代わりに[H]と書きこむと強制的にその場所に図や表を挿入する
\usepackage{bm}
\usepackage{amsmath}

%%%%%%%%%%%%%%%%%%%%%%%%%%%%%%%%%%%%%%%%%%%%%%%%%
%%newcommand群
%%%%%%%%%%%%%%%%%%%%%%%%%%%%%%%%%%%%%%%%%%%%%%%%%
\newcommand{\argmax}{\mathop{\rm arg~max}\limits}
\newcommand{\argmin}{\mathop{\rm arg~min}\limits}

\newcommand{\EE}{{\mathbb E}} % 期待値のE (追加)
\newcommand{\cN}{{\cal N}}
% \newcommand{\a}{\bm{a}}
% \newcommand{\b}{\bm{b}}
% \newcommand{\c}{\bm{c}}
% \newcommand{\d}{\bm{d}}
% \newcommand{\e}{\bm{e}}
% \newcommand{\f}{\bm{f}}
% \newcommand{\g}{\bm{g}}
% \newcommand{\h}{\bm{h}}
% \newcommand{\i}{\bm{i}}
% \newcommand{\j}{\bm{j}}
% \newcommand{\k}{\bm{k}}
% \newcommand{\l}{\bm{l}}
% \newcommand{\m}{\bm{m}}
% \newcommand{\n}{\bm{n}}
% \newcommand{\o}{\bm{o}}
% \newcommand{\p}{\bm{p}}
% \newcommand{\q}{\bm{q}}
% \newcommand{\r}{\bm{r}}
% \newcommand{\s}{\bm{s}}
% \newcommand{\t}{\bm{t}}
% \newcommand{\u}{\bm{u}}
% \newcommand{\v}{\bm{v}}
% \newcommand{\w}{\bm{w}}
% \newcommand{\x}{\bm{x}}
% \newcommand{\y}{\bm{y}}
% \newcommand{\z}{\bm{z}}
\newcommand{\ul}{\underline}
\newcommand{\us}{\underset}

\newcommand{\zl}{\rightarrow}
\newcommand{\zh}{\leftarrow}

%%%%%%%%%%%%%%%%%%%%%%%%%%%%%%%%%%%%%%%%%%%%%%%%%
%%追加
%%%%%%%%%%%%%%%%%%%%%%%%%%%%%%%%%%%%%%%%%%%%%%%%%
\renewcommand{\algorithmicrequire}{\textbf{Input:}}


%%%%%%%%%%%%%%%%%%%%%%%%%%%%%%%%%%%%%%%%%%%%%%%%%
%%本文
%%%%%%%%%%%%%%%%%%%%%%%%%%%%%%%%%%%%%%%%%%%%%%%%%
\title[]{Gaussian Process Optimization in the Bandit Setting:\\No Regret and Experimental Design}
\subtitle{}
\author[]{Niranjan Srinivas, Andreas Krause$^{\dag}$, Sham Kakade$^{\dag\dag}$, Matthias Seeger$^{\dag\dag\dag}$
			\\ \vspace{20pt} \large{\centering{ICML 2020 Test of time Award} \vspace{9pt}}}
% \date[\today]{\today} %日付を表示
\date[\today]{} %日付を非表示
\institute[]{$\dag$: California Institute of Technology\\
$\dag\dag$: University of Pennsylvania\\
$\dag\dag\dag$: Saarland University}
% [..]に省略名が書ける

%目次スライド
\AtBeginSection[noframenumbering]{
    \begin{frame}[noframenumbering]{Next Section}
	\tableofcontents[currentsection, hidesubsections]%目次本体
	\thispagestyle{empty}%ヘッダーフッター表示なし
	\end{frame}
}

\begin{document}

%-------------------

%タイトル
\begin{frame}[plain, noframenumbering]
\maketitle%タイトル本体
\thispagestyle{empty}%ヘッダーフッター表示なし
% \addtocounter{framenumber}{-1}%ページ数カウンタしない
\vspace{-6pt}
\centering{Presenter 石倉 雅紀}
\end{frame}

%-------------------


%目次
\begin{frame}[noframenumbering]{目次}
	\tableofcontents[hideallsubsections]
	\thispagestyle{empty} %ページ番号なし
\end{frame}

%-------------------
\section{はじめに}
%-------------------

\subsection{背景}
\begin{frame}{\insertsubsection}
	\begin{itemize}
		\item 目的関数に対する最適変数探索問題として定式化できる問題を考える
		\begin{itemize}
			\item 耐久性の高いロボットの開発
		\end{itemize}
	\end{itemize}
	\vspace{10pt}
	\begin{center}
		\includegraphics[width=0.80\textwidth]{./Fig/Robot.pdf}
	\end{center}
\end{frame}

%-------------------



\subsection{ブラックボックス関数}
\begin{frame}{\insertsubsection}
	\begin{itemize}
		\item ブラックボックス関数 $f$
		\begin{align*}
			y_i = f(\bm x_i) + \varepsilon_i
		\end{align*}
		\begin{center}
			入力:$\bm x_i$,\quad
			出力:$y_i$,\quad
			誤差:$\varepsilon_i$
		\end{center}
	\end{itemize}

	\vspace{5pt}
	\begin{center}
		\includegraphics[width=0.50\textwidth]{./Fig/blackbox.pdf}
	\end{center}
	\vspace{5pt}
	
	\begin{itemize}
		\item ブラックボックス関数$f$
		\begin{itemize}
			\item 具体的な形状が不明
		\end{itemize}
		\item ブラックボックス関数を扱う実問題も多く存在する
	\end{itemize}
\end{frame}

%-------------------



\subsection{ベイズ最適化}
\begin{frame}{\insertsubsection}
%		\begin{itemize}
%			\item ベイズ最適化は,確率論的アプローチを用いた最適化手法である
%			\item そのため,不確実性を含む問題を解決するために用いられていることが多い
%			\item そのような問題を扱うために,事前分布を用いて事後分布を求めることができる
%			\item 論文中では,このようなベイズ最適化を用いて,ガウス過程を用いた最適化問題を解決するアルゴリズムを提案している.
%			\item また,そのアルゴリズムがバンディット設定でも有効であることを示している.
%		\end{itemize}

	\begin{itemize}
		\item ブラックボックス関数を最適化する問題を考える
		\vspace{2pt}
		\item \textcolor{myorange}{できるだけ少ない関数評価回数で最適値を見つけたい}
		\begin{itemize}
			\item 特に関数値を得るコストが高い場合
		\end{itemize}
		\vspace{1pt}
		\item 有効な手法として\textcolor{myorange}{ベイズ最適化}がある
	\end{itemize}
	\vspace{10pt}
	\begin{center}
		\includegraphics[width=0.80\textwidth]{./Fig/BO.pdf}
	\end{center}

\end{frame}

%-------------------

	% \subsection{バンディット問題}
	% \begin{frame}{\insertsubsection}
	% 	\begin{itemize}
	% 		\item 序論として,バンディット問題について説明する(前提知識の提供)
	% 		\begin{itemize}
	% 			\item バンディット設定とは,決定をする際に報酬を得ることができるが,そのためには,その決定をする前にその詳細を知ることができない設定を指す
	% 			\item このような設定では,報酬を最大化することを目的として,決定をする場所を最適に選択することが重要になる
	% 			\item バンディット問題の具体例:投資をする場合や,商品を購入する場合など.このような場面では,最適な決定をすることで,最大の報酬を得ることができるため,重要である.
	% 		\end{itemize}
	% 	\end{itemize}
	% \end{frame}


%-------------------

\subsection{本論文の貢献}

\begin{frame}{\insertsubsection}
	\vspace{20pt}
	\begin{block}{貢献}
		\begin{itemize}
			\vspace{5pt}
			\item ガウス過程を用いたベイズ最適化の獲得関数を提案
			\begin{itemize}
				\vspace{3pt}
				\item Gaussian Process Upper Confidence Bound(GP-UCB)
			\end{itemize}
			\vspace{6pt}
			\item 実データ実験において優れた性能を発揮
			\vspace{5pt}
		\end{itemize}
	\end{block}


		% \begin{itemize}
		% 	\item 計算機科学や統計学において,最適化問題は非常に重要であることを強調する
		% 	\item ガウス過程を用いた最適化は,複雑で非線形な関数を扱う場合に有用であることを紹介する
		% 	\item バンディット設定において,最適な決定をすることが重要であることを紹介する
		% 	\item 今までにバンディット設定でのガウス過程最適化については,リグレット制限が証明されていなかった
		% 	\item 本論文では,バンディット設定におけるガウス過程最適化において,リグレット制限を証明することで,最適な決定をするアルゴリズムを提案する
		% \end{itemize}

\end{frame}
%-------------------

%以降\sectionごとに目次表示
\section{ベイズ最適化}
%-------------------
% \subsection{ブラックボックス関数の定式化}
% \begin{frame}{\insertsubsection}
% 	\begin{itemize}
% 		\item[$\ast$] 本スライドでは,一章で述べたブラックボックス関数を定式化して説明する
% 		\item ブラックボックス関数
% 		\begin{align*}
% 			y_i = f(\bm x_i) + \varepsilon_i
% 		\end{align*}
% 		\begin{center}
% 			入力:$\bm x_i \in \mathcal{X} \subseteq \mathbb{R}^d$,\quad
% 			出力:$y_i \in \mathbb{R}$,\quad
% 			誤差:$\varepsilon_i \in \mathbb{R}$
% 		\end{center}
% 	\end{itemize}

% 	\begin{center}
% 		\includegraphics[width=0.80\textwidth]{./Fig/blackbox.pdf}
% 	\end{center}

% \end{frame}

%-------------------

\subsection{問題設定}
\begin{frame}{\insertsubsection}
	\begin{itemize}
		\vspace{20pt}
		\item 候補入力$\mathcal{X} = \{ \bm x_1, \ldots, \bm x_n \}$が与えられている
		\vspace{9pt}
		\item 関数$f$を評価して出力$y_i = f(\bm x_i)$を得るにはコストがかかる
		\vspace{9pt}
		\item \textcolor{myorange}{できるだけ少ないコスト}で\\
  \vspace{2pt}\textcolor{myorange}{ブラックボックス関数$f$を最大化するパラメータ$\bm x$}を求めたい
		\begin{align*}
			\bm x^* = \arg \max_{\bm x \in \mathcal{X}} ~ f(\bm x)
		\end{align*}
	\end{itemize}
\end{frame}

%-------------------

\subsection{ガウス過程回帰}
\begin{frame}{\insertsubsection}
	\begin{block}{ガウス過程}
		任意の入力$\{ \bm x_1, \ldots, \bm x_n \}$に対して\\
		$\{f(\bm x_1), \ldots, f(\bm x_n)\}$がn次元正規分布に従うなら$f$はガウス過程に従う
	\end{block}

\begin{itemize}
	\item 関数$f$がガウス過程に従うことを$f(\bm x) \sim \mathcal{GP}(\mu(\bm x), k(\bm x, \bm x'))$
	\begin{itemize}
		\item 平均関数: $\mu(\bm x) = \mathbb{E}[f(\bm x)]$
		\vspace{1pt}
		\item 共分散関数: $k(\bm x, \bm x') = \mathbb{E}[(f(\bm x) - \mu(\bm x))(f(\bm x') - \mu(\bm x'))]$
	\end{itemize}
	\end{itemize}
\begin{block}{ガウス過程回帰}
	\begin{itemize}
		\item $f$の事前分布がガウス過程であると仮定: $f(\bm x) \sim \mathcal{GP}( 0, k(\bm x, \bm x'))$
		\item 観測データ$(X, \bm y) = \{(\bm x_i, y_i)\}_{i = 1}^t = \mathcal{A}$が与えられた下での事後分布
	\end{itemize}
	\vspace{-0.2cm}
	\begin{align*}
		f({\bm x}_\ast) \mid \mathcal{A}
			\sim
			N \left(
			%\underbrace{
			\bm k_{\mathcal{A}, {\bm x}_\ast   }^\top
			K_{\mathcal{A} \mathcal{A}}^{-1}
			{\bm y},%}_{\text{posterior mean}}
			%\underbrace{
			k_{{\bm x}_\ast {\bm x}_\ast}
			-
			\bm k_{\mathcal{A} , {\bm x}_\ast}^\top
			K_{\mathcal{A} \mathcal{A}}^{-1}
			\bm k_{\mathcal{A} , {\bm x}_\ast}%}_{\text{posterior variance}}
			\right)
	\end{align*}
	\vspace{-1cm}

	\begin{onlyenv}<1>

		\begin{tiny}
			\begin{align*}
				\hspace*{-15mm}
				K_{\mathcal{A} \mathcal{A}}
				=
				\begin{pmatrix}
					k(\bm x_{ 1}, \bm x_{ 1}) &
					\cdots &
					k(\bm x_{ 1}, \bm x_{ t}) \\
					\vdots &
					\ddots &
					\vdots \\ 	 
					k(\bm x_{ t}, \bm x_{ 1}) &
					\cdots &
					k(\bm x_{ t}, \bm x_{t}) 
				\end{pmatrix},
				\bm k_{\mathcal{A} ,{\bm x}_\ast}
				=
				\begin{pmatrix}
					k(\bm x_{ 1}, \bm x_\ast) \\
					\vdots \\
					k(\bm x_{ t}, \bm x_\ast) 
				\end{pmatrix},		 %
				k_{{\bm x}_\ast, {\bm x}_\ast}
				=
				k(\bm x_{\ast}, \bm x_{\ast})
			\end{align*}
		\end{tiny}
	\end{onlyenv}

	\begin{onlyenv}<2>
		\begin{itemize}
			\vspace{18pt}
			\centering{\item[$\Rightarrow$] \textcolor{myorange}{解析的に}事後分布のパラメータが求まる}
			\vspace{18pt}
		\end{itemize}
	\end{onlyenv}
\end{block}
\end{frame}

%-------------------


\subsection{ガウス過程を用いたベイズ最適化}
\begin{frame}{\insertsubsection}
	\begin{itemize}
		\vspace{10pt}
		\item $f(\bm x) \sim \mathcal{GP}(0, k(\bm x, \bm x'))$を仮定する
	\end{itemize}
	\begin{center}
		\includegraphics[width=0.80\textwidth]{./Fig/GP.pdf}
	\end{center}
	\begin{itemize}
		\item 訓練データ$\mathcal{D} = (X, \bm y) = \left\{ (\bm x_i, y_i) \right\}_{i=1}^t$に基づき予測モデルを更新
		\vspace{2pt}
		\item 予測モデルに基づき\textcolor{myorange}{次の点を選ぶ指標を用いて最適な点}を次に観測
		\vspace{2pt}
		\item 観測した$(x_{\rm next}, y_{\rm next})$を訓練データに追加し再び予測モデルを更新
	\end{itemize}




\end{frame}

%-------------------


\subsection{獲得関数}
\begin{frame}{\insertsubsection}
	\begin{block}{獲得関数}
		探索する点を選択するために使用される関数 % \iffalse$\bm x_t$\fi, \iffalse$\alpha$\fi
	\end{block}
	\begin{itemize}
		\item 獲得関数を$\alpha(\bm x)$とすると次に観測する点$\bm x_t$は以下のように決定
	\end{itemize}
	\begin{align*}
		\bm x_t = \arg \max_{\bm x} ~ \alpha(\bm x)
	\end{align*}
	\begin{block}{獲得関数の設計指針}
		\textcolor{myorange}{活用と探索のバランス}を考慮して設計する
	\end{block}
	\begin{itemize}
		\item 活用: 最適解がありそうな点を探索すること
		\begin{itemize}
			\item 活用重視のとき局所解に陥りやすい
		\end{itemize}
		\vspace{3pt}
		\item 探索: 未知の領域を探索し新しい知見を得ること
		\begin{itemize}
			\item 探索重視のとき最適解にたどり着くまでに時間を要する
		\end{itemize}
	\end{itemize}

\end{frame}

%-------------------


\subsection{従来の獲得関数}
\begin{frame}{\insertsubsection}
	\begin{block}{mean only}
		\begin{align*}
			\bm x_t = \arg \max_{\bm x} ~ \textcolor{myorange}{\mu}_{t-1}(\bm x)
		\end{align*}
	\end{block}
	\begin{itemize}
		\item 予測平均が最大となる$\bm x$を選択
		\begin{itemize}
			\item[$\Rightarrow$] \textcolor{myorange}{活用}中心
		\end{itemize}
	\end{itemize}
	\begin{block}{variance only}
		\begin{align*}
			\bm x_t = \arg \max_{\bm x} ~ \textcolor{myorange}{\sigma}_{t-1}^{\textcolor{myorange}{2}}(\bm x)
		\end{align*}
	\end{block}
	\begin{itemize}
		\item 予測分散が最大となる$\bm x$を選択
		\begin{itemize}
			\item[$\Rightarrow$] \textcolor{myorange}{探索}中心
		\end{itemize}
	\end{itemize}
\end{frame}
%-------------------

\section{GP-UCB}
%-------------------

\subsection{提案手法}

\begin{frame}{\insertsubsection}
	\begin{itemize}
		\item 探索・活用に特化した手法では最適化がうまくいかない場合が多い
		\begin{itemize}
			\item[$\Rightarrow$] \textcolor{myorange}{探索と活用のバランスが取れるような手法}を提案
		\end{itemize}
	\end{itemize}
	\vspace{5pt}
	\begin{block}{GP-UCB: Gaussian Process Upper Confidence Bound}
		\begin{align*}
			\bm x_t = \arg \max_{\bm x \in \mathcal{D}} ~ \mu_{t-1}(\bm x) + \beta_t^{1/2} \sigma_{t-1}(\bm x).
		\end{align*}
	\end{block}
	\vspace{10pt}
	\begin{itemize}
		\item $\beta_t$は,探索の度合いを表す定数
		\begin{itemize}
			\item $\beta_t$が小さい $\Rightarrow$ 予測平均の比重が大きくなる $\Rightarrow$ 活用重視
			\vspace{1pt}
			\item $\beta_t$が大きい $\Rightarrow$ 予測分散の比重が大きくなる $\Rightarrow$ 探索重視
		\end{itemize}
	\end{itemize}

\end{frame}
%-------------------

\subsection{GP-UCBのアルゴリズム}
\begin{frame}{\insertsubsection}
\begin{algorithm}[H]
	\caption{The GP-UCB algorithm.}
	\label{alg:gpucb}
	\begin{algorithmic}
		\Require Input space $\mathcal{D}$; GP Prior $\mu_0 = 0$, $\sigma_0$, $k$
		\For{\texttt{$t = 1, 2,\ldots $}}			\State \textcolor{myorange}{Choose $\bm x_t = \arg \max_{\bm x \in \mathcal{D}} ~ \mu_{t-1}(\bm x) + \sqrt{\beta_t} \sigma_{t-1}(\bm x)$}
			\State Sample $y_t \gets f(\bm x_t) + \varepsilon_t$
			\State Perform Bayesian update to obtaion $\mu_t$ and $\sigma_t$
		\EndFor
	\end{algorithmic}
\end{algorithm}




\end{frame}

%-------------------

\section{実験}
%-------------------



\subsection{実験手法}
\begin{frame}{\insertsubsection}
	\begin{itemize}
		\item 実験手法
		\begin{itemize}
			\vspace{5pt}
			\item variance only
			\vspace{2pt}
			\item mean only
			\vspace{2pt}
			\item Expected Improvement(EI)
			\begin{itemize}
				\item 目的関数の改善量の期待値を最大化する指標
			\end{itemize}
			\vspace{2pt}
			\item Most Probable Improvement(MPI)
			\begin{itemize}
				\item 目的関数が改善する確率が高い点を選択する指標
			\end{itemize}
			\vspace{2pt}
			\item \textcolor{myorange}{GP-UCB}
		\end{itemize}
		\vspace{15pt}
		\item \textcolor{myorange}{Mean Average Regret} を用いた評価
		\begin{itemize}
			\item cumulative regret: $R_t = \sum_{t=1}^T\{f(\bm x^*) - f(\bm x^t)\}$
			\begin{itemize}
				\vspace{1.5pt}
				\item $\mbox{評価対象の関数}f\mbox{の最大値}f(\bm x^*)$
				\vspace{1.5pt}
				\item $\mbox{時点}t\mbox{において選ばれた関数値}f(\bm x^t)$
			\end{itemize}
			\vspace{2pt}
			\item Average Regret: cumulative regretをイテレーション数$T$で割ったもの
			\vspace{2pt}
			\item \textcolor{myorange}{Mean Average Regret} : 全試行のAverage Regretの平均
		\end{itemize}
	\end{itemize}
\end{frame}

%-------------------
\subsection{実験対象データ}
\begin{frame}{\insertsubsection}
	\begin{enumerate}
		\item 合成データ
		\begin{itemize}
			\item $\mathcal{D} \in [0,1]$の範囲を一様に1000点で区切った点を候補点とする
			\item $T = 1000$, $\sigma^2 = 0.025$, $\delta = 0.1$
			\item 30回試行
		\end{itemize}
		\vspace{5pt}
		\item 温度データ
		\begin{itemize}
			\item 46個のセンサーから1分間隔で5日以上計測された気温
			\item $T = 46$, $\sigma^2 = 0.5$, $\delta = 0.1$
			\item 2000回試行
		\end{itemize}
		\vspace{5pt}
		\item 交通データ
		\begin{itemize}
			\item 357個のセンサーから1ヶ月間午前6時から11時に通過する車の速度
			\item $T = 357$, $\sigma^2 = 4.78$, $\delta = 0.1$
			\item 900回試行
		\end{itemize}
	\end{enumerate}
\end{frame}

%-------------------

\subsection{実験結果}

\begin{frame}{\insertsubsection}
	\begin{center}
		\includegraphics[width=0.90\textwidth]{./Fig/Figure4.pdf}
	\end{center}
	\begin{itemize}
		\item すべての実験結果においてGP-UCBが小さいリグレットで最適化を行なっている
		\item \textcolor{myorange}{既存手法のEI,MPIと比較して同等以上の性能を示した}
	\end{itemize}

\end{frame}

%-------------------
\section{まとめ}

\begin{frame}{\insertsection}
	\begin{itemize}
		\item GP-UCBという\textcolor{myorange}{ベイズ最適化の獲得関数を新たに提案}
		\begin{itemize}
			\vspace{2pt}
			\item GP-UCBは活用と探索を両立する獲得関数
		\end{itemize}
		\vspace{10pt}
		\item GP-UCBは\textcolor{myorange}{既存の手法と同等以上の性能}を示した
		\begin{itemize}
			\vspace{2pt}
			\item Mean Average Regretの観点で実データと合成データにおける性能を発揮
		\end{itemize}
		% \item We analyze GP-UCB, an intuitive algorithm for GP optimization, when the function is either sampled from a known GP, or has low RKHS norm.
		% \item We bound the cumulative regret for GP-UCB in terms of the information gain due to sampling, establishing a novel connection between experimental design and GP optimization.
		% \item By bounding the information gain for popular classes of kernels, we establish sublinear regret bounds for GP optimization for the first time. Our bounds depend on kernel choice and parameters in a fine-grained fashion.
		% \item We evaluate GP-UCB on sensor network data, demonstrating that it compares favorably to ex- isting algorithms for GP optimization.
	\end{itemize}

\end{frame}

%-------------------

\setbeamertemplate{footline}{} % フッターを消す
\begin{frame}[noframenumbering]{参考文献}
	\begin{thebibliography}{}
		\setbeamertemplate{bibliography item}[book]
		\bibitem[Mockus, 1989]{Mockus1989}
		Mockus, J. Bayesian Approach to Global Optimization. Kluwer Academic Publishers, 1989.

		\setbeamertemplate{bibliography item}[inbook]
		\bibitem[Mockus et~al., 1978]{Mockus1978}
		Mockus, J., Tiesis, V., and Zilinskas, A. Toward Global Optimization, volume 2, chapter Bayesian Methods for Seeking the Extremum, pp. 117–128. 1978.

		\setbeamertemplate{bibliography item}[book]
		\bibitem[Williams and Rasmussen, 2006]{Williams2006}
		C.K. Williams and C.E. Rasmussen, Gaussian processes for machine learning, vol.2, MIT press Cambridge, MA, 2006.		\end{thebibliography}		
\end{frame}

%-------------------

% \subsection{まとめ2}
% \begin{frame}{\insertsubsection}
% 	\begin{itemize}
% 		\item テスト
% 	\end{itemize}

% \end{frame}

% %-------------------

\end{document}








